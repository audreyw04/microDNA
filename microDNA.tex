\documentclass[11pt, letterpaper]{article}
\usepackage[margin=0.7in]{geometry}
\usepackage{amsmath}
\usepackage{graphicx}
\usepackage{booktabs}


\begin{document}

\title{Micro DNA}
\author{Audrey Weaver}
\maketitle

\section{Introduction}

MicroDNA refers to small extrachromosomal circular DNA elements. These circular
chunks of DNA typically range from 100-1000 base pairs in length. They have been 
detected in in eukariotic cells and are thought to be the result of DNA replication,
transcriptional variations, or other repair processes. 

In this project, we used an algorithm to detect the circular fragments of microDNA. 
Soft clips were identified and isolated from the BAM file. These are sequences which 
do not fully align with the sequence of the chromosome, indicating that they may be 
circular junctions. Reads were then filtered by length and grouped by position and 
sequence to detect potential junctions. Reliant on the identification of repeated 
sequences and soft clips from a BAM file, the algorithm then used smith-waterman 
to align the potentially overlapping reads on either side of the circle and detect 
the microhomology between them.  

A confidence score was then calculated to weigh all factors contributing to the 
likelihood that the sequence is in fact a circle. Similar to the score given by 
the SequenceMatcher package, we used Smith-Waterman which had already beem developed 
for an assignment to match all of the microhomology sequences and give them a score. 
We were inspired by the scores given by SequenceMatcher and wanted to use our own 
algorithm while still being able to have a comparable score for each read. Mainly 
focused on sequence similarity and their junction proximity within the genome, our 
goal was to prioritize biologically realistic microDNA circles and validate them. 

\section{Results}

A total of 23,640 soft-clipped reads were extracted from the aligned BAM file using a 
minimum clip length of 12 base pairs. These reads were split into start- and end-clipped 
groups and compared to identify potential microDNA circles. Junction pairs were required 
to be on the same chromosome, within 1,000 base pairs of each other, and have matching or 
highly similar clipped sequences as determined by local sequence alignment. While all data
in the BAM file studied was originating from the same chromosome, this is something that 
must be considered for future use of this algorithm so we implemented considerations for 
the chromosome the sequences were found on. 

After scoring each candidate using a confidence metric that weights microhomology against 
circle length, 948 distinct candidate circles were identified. We applied additional 
validation criteria requiring at least one supporting read on both the start and end of 
the circle, at least two total reads, and a minimum confidence score of 0.3. This reduced 
the candidate set to a smaller group of high-confidence circles, summarized in Table 1.

\begin{table}
    \centering
    \caption{Top 5 Validated microDNA Circles (Escaped for LaTeX)}
    \label{tab:validated-circles}
    \resizebox{\textwidth}{!}{%
    \begin{tabular}{lrrllrrr}
    \toprule
                Chromosome &     Start &       End &    Start Seq &      End Seq & Start Reads & End Reads & Total \\
    \midrule
    \texttt{NC\_000001.10} &  17912842 &  17912938 & TTTTTAAACATC & TTTTTAAACATC &          23 &        40 &    63 \\
    \texttt{NC\_000001.10} & 121485155 & 121485216 & GAATATCCACTT & TGAATATCCACT &           3 &         1 &     4 \\
    \texttt{NC\_000001.10} & 121485177 & 121485188 & TGGAATTTGCAA & TGGAATTTGCAA &           2 &         1 &     3 \\
    \texttt{NC\_000001.10} & 121485185 & 121485358 & GTTTGTAAAGTC & TTTGTAAAGTCT &           2 &         1 &     3 \\
    \texttt{NC\_000001.10} & 121485356 & 121485378 & TTTTGTGGAATT & TTTTTGTGGAAN &           2 &         1 &     3 \\
    \bottomrule
    \end{tabular}%
    }
\end{table}
    
    
    

Most validated circles were short, with distances under 200 base pairs and 
perfect or near-perfect sequence alignment between junction reads. One outlier 
(chr1:17,912,842–17,912,938) had 63 total supporting reads and a repeated 
clipped sequence (TTTTTAAACATC), suggesting possible over-amplification or 
origin from a low-complexity region. 

Despite the unusually high number of supporting reads and the repetitive nature 
of its clipped sequence, this candidate passed all validation criteria. 
Specifically, it had soft-clipped reads on both ends of the predicted circle 
junction, each with the same clipped sequence (TTTTTAAACATC). It also had a total 
support count of 63 reads with 23 at the start, 40 at the end. The clipped sequences 
showed perfect microhomology, and the base pair distance between junctions was only 
96 bp, resulting in a high confidence score. Strong read support, matching sequences, 
and short circle length were all features indicating that the signal is consistent and 
not an result of poor alignment or random clipping.

\section{Methods}

To identify potential microDNA circles, we developed a Python pipeline that 
detects and scores soft-clipped reads from aligned sequencing data. The input 
is a coordinate-sorted BAM file with its index. Soft-clipped reads were 
extracted using pysam by checking for soft clipping at either the 5' or 3' 
end of the read (CIGAR notation S). Reads were filtered based on a minimum 
soft clip length, which we set at 12 bp for most of the analysis to avoid 
short, possibly noisy clips.

Each clipped read was recorded along with its position, CIGAR string, and 
clipped sequence. Reads were separated into “start-clipped” and “end-clipped” 
groups based on the side of the alignment where the clip occurred. To identify 
candidate circles, we paired clipped reads on the chromosome that were within 
1,000 bp of each other and had similar clipped sequences. This was chosen due to 
the assumption that most microDNA circles to not excede 1,000 base pairs in length. 

To measure sequence similarity between clipped reads, we implemented a Smith-Waterman 
local alignment algorithm. The alignment score was normalized and used to calculate 
microhomology between each start-end pair. We then applied a confidence score that 
weighted sequence similarity against the genomic distance between positions. Candidates 
with higher similarity and shorter distances received higher confidence scores.

\begin{equation}
    \text{confidence} = \text{microhomology} \times \left(1 - \frac{\text{distance}}{\text{MAX\_DISTANCE}}\right)
    \end{equation}

The final output includes all paired junctions with their positions, clipped sequences, 
distance, microhomology score, and calculated confidence. For validation, we counted the 
number of supporting clipped reads for each junction and considered circles with at least 
one read on both ends and total support of two or more to be valid. Confidence scores and 
clipped sequence quality were used to further analyze results.

All scripts were written to include potential variability in inputs and outputs according 
to what was changed throughout development of this algorithm. Minimum length of clip and 
possibility for validating multiple start positions at once allowed for trial with the 
BAM file and the algorithm and were kept for potential future users. 


\subsection{Reproducibility}
To replicate these experiments, clone the repository and then run the
following commands from the root directory of the repository.

\begin{verbatim}
$ git clone https://github.com/audreyw04/microDNA.git
$ python3 detecting.py --bam ./data/SRR413984.sorted.NC_000001.10.bam \ 
                        --min_clip 12 \ 
                        --out clips_12bp.tsv

$ python3 validation.py --circles microdna_circles.tsv \
                        --reads clips_12bp.tsv \
                        --start_positions 121485177 121485185 121373488 17912842 121485155 
                        121485356 121485363 121485210 121484883 121484600 121485000 
                        121484512
\end{verbatim}

\subsection{References}
\begin{thebibliography}{9}

    \bibitem{openai2024}
    OpenAI. (2024). \textit{ChatGPT} (April 4 version) [Large language model]. https://chat.openai.com/
    
    \end{thebibliography}

    ChatGPT was used to assist with refining ideas, debugging code and editing for this assignment. 

\end{document}
